\input{Formatting/Bylaws-Preamble.tex}

\title[General]

    \article[Mission Statement]
        \section The Associated Students of Walla Walla University (ASWWU) is the largest student-run organization on campus, compromised of students, faculty, and staff with this mission: Empower the student, unify the campus, and serve the world.

    \article[Meetings of the Associated Students]
        \section[Time and Place]
            \subsection The Associated Students of Walla Walla University (ASWWU) is the largest student-run organization on campus, compromised of students, staff, and faculty with the mission to enrich student quality of life.
                \paragraph The ASWWU Introduction, to occur during autumn quarter, to familiarize the members with their governing body.
                \paragraph The ASWWU Week of Worship, to occur during the second week of winter quarter, to give the Spiritual Vice President an opportunity to impact the campus spiritually.
            \subsection The ASWWU Assemblies are to occur during CommUnity.
            \subsection The ASWWU Assembly agendas, once approved by the ASWWU President, must be submitted for review to the CommUnity Committee two (2) meetings prior to the assembly.
        \section[Procedure]
            \subsection At the Introduction Assembly, there shall be an update from each of the ASWWU administrative departments on their mission/purpose and their continuing/new projects.
            \subsection All such assemblies shall be publicized in \textit{The Collegian} and on the ASWWWU Website with a tentative agenda for at least seven (7) business days prior to assembly.
        \section[Confab]
            \subsection Confab is a student-led conference that provides a forum of direct communication between the students and the administrators.
            \subsection Confab is to occur at least once per quarter on topics of interest to the student body.
                \paragraph The ASWWU President or his/her appointee is responsible for determining topics.

    \article[Membership]
        \section[Composition]
            \subsection Any undergraduate student currently enrolled at Walla Walla University and any faculty or staff member currently employed by Walla Walla University may be a member of the ASWWU.
            \subsection Any undergraduate student enrolled in six (6) or more credits per quarter will be automatically charged ASWWU dues; if a student is taking fewer than six (6) credits, he/she may choose to pay dues in order to become a member of the ASWWU.
            \subsection Undergraduate students enrolled in either the Adventist Colleges Abroad (ACA) or Student Missionary (SM) programs are not automatically charged ASWWU dues. A student in the ACA or SM programs may choose to pay dues in order to become a member of the ASWWU.
            \subsection Graduate students are not automatically charged ASWWU dues. Any graduate student may choose to pay dues in order to become a member of the ASWWU.
            \subsection Any faculty/staff member not automatically charged ASWWU dues may choose to pay dues in order to become a member of the ASWWU.
        \section[Dues]
            \subsection Membership dues for the ASWWU shall be an amount expressed in dollars per quarter and shall be charged for the autumn, winter, and spring quarters.
            \subsection The ASWWU dues shall be a proportion of the full-time equivalent (FTE).
            \subsection Graduate students who choose to pay ASWWU dues will be charged the same rate as undergraduate students.
            \subsection Any changes in dues must be presented by the President to Senate and passed by a two-thirds roll call vote, whereupon the new amount of the membership dues shall be forwarded to Student Financial Services for approval. Student Financial Services shall charge the membership dues and shall be responsible for entering the amount of the membership dues in the Bulletin.
            \subsection Any change in the amount of the membership fees shall take effect upon the following autumn quarter and shall remain in effect until further action by the Senate.
            \subsection Changes in tuition for Walla Walla University are to be automatically reflected in membership dues.

    \article[External Relations]
        \section[Adventist Intercollegiate Association]
            \subsection[Purpose]
                \paragraph The Adventist Intercollegiate Association (AIA) is a self-governing body with membership consisting of the student governments from all North American Seventh-day Adventist (SDA) colleges and universities. Its purpose is to, first, provide a forum for the discussion of shared collegiate concerns and a united voice in expressing these concerns to those in positions to bring results, and, second, to provide workshops for the exchange of ideas between colleges in order to foster activities. AIA's activities occur at an annual convention held in early spring.
            \subsection[Relations with AIA]
                \paragraph The ASWWU is a member of AIA, and as such it is responsible for paying annual dues as specified in the AIA Bylaws.
            \subsection[Delegate Selection]
                \paragraph The delegates to AIA's annual convention are selected by the President and are ratified by Senate.
                \paragraph In order to properly represent the entire ASWWU student government, delegates to the AIA convention must be selected to represent each of the following ASWWU branches:
                    \subparagraph \textbf{Executive:} Outgoing and incoming Presidents
                    \subparagraph \textbf{Legislative:} Outgoing and incoming Executive Vice Presidents
                    \subparagraph \textbf{Social:} Outgoing and incoming Social Vice Presidents
                    \subparagraph \textbf{Spiritual:} Outgoing and incoming Spiritual Vice Presidents
                \paragraph Exceptions may be approved by Senate when circumstances warrant and when requested by the President.
                \paragraph A complete list of proposed delegates must be submitted to Senate for ratification by the next full week after ASWWU elections.
                \paragraph The ASWWU Sponsor must attend the convention with the ASWWU delegates, but in cases where this is not possible, he/she should appoint a faculty or staff member as the trip advisor.
            \subsection[Budget]
                \paragraph A detailed travel budget is to be presented to Senate by the President, including costs for airfare, ground transportation, and any other costs which will be incurred at the convention. This budget is to be submitted for review by the next full week following ASWWU General Elections.
                \paragraph The Senate must authorize attendance of ASWWU delegates to conferences if the total cost of the event is more than \$250.
            \subsection[Hosting]
                \paragraph On years when WWU is selected as the host site, the ASWWU President, in consultation with the AIA President, is to draw up a detailed list of WWU's hosting responsibilities.
                \paragraph A report including the responsibilities, plan, and budget is to be made to Senate no later than 1 February, and a final report is to be made on the same at the first meeting in March.
        \section[Representation]
            \subsection If consent for representation occurs, regardless of funding, upon participating in event, the delegates must present a report to the President or EVP. A report may be made to Senate at the request of the President.

    \article[Records]
        \section[Constitution and Bylaws]
            \subsection At least two copies of the Constitution and Bylaws shall be maintained on reserve in the Peterson Memorial Library at all times.
            \subsection The ASWWU Governing Documents shall be kept up to date and available within two weeks of any changes, including any new Senate governance legislation. Responsibility of updating the Governing Documents shall fall upon the Parliamentarian, assisted as needed by the Senate Secretary.
        \section[Proceedings]
            \subsection A complete copy of the proceedings of the ASWWU and the Senate shall be made available in the ASWWU offices. Proceedings of the ASWWU will be an accumulation of all minutes except for those of Cabinet.
        \section[Minutes]
            \subsection[Definition:] The minutes are the official transcript of a meeting as described in the current edition of \textit{Robert's Rules of Order} for parliamentary procedure.
            \subsection[Scope:] Minutes shall be taken for Senate and Executive Committee.
            \subsection[Duties:]
                \paragraph[Senate Secretary]
                    \subparagraph The Senate Secretary shall be responsible for recording minutes for Senate and Executive Committee.
                    \subparagraph Should the Senate Secretary be unable to attend such a meeting, a temporary secretary shall be appointed prior to the meeting by the Executive Vice President. Should the Senate Secretary not attend a meeting without making prior arrangements per above, the meeting shall be cancelled and considered to have no standing until such time as a secretary is present.
            \subsection[Parliamentarian:] The Chief Justice shall be charged with overseeing compliance for this article. The Chief Justice shall keep files of all such minutes, forward them to the President, and shall make them publicly available upon request of any member of the ASWWU.
        \section[Financial Records]
            \subsection A copy of the ASWWU Financial Records shall be made available to all members of the ASWWU in the ASWWU offices.
        \section[ASWWU Digital Senate Archive]
            \subsection[Content:] The ASWWU Digital Senate Archive (ADSA) will contain all official documents of the ASWWU and shall be updated with official copies of the documents required in Title 1, Article V, Section 3.
            \subsection[Preservation and development:] The Chief Justice will be responsible for the integrity of the documents in the archive. The Chief Justice will also be responsible for coordinating with the Webmaster to ensure the update of the ADSA.
            \subsection[Availability:] The Webmaster will be responsible for ensuring the availability of the online archives as well as facilitating the ongoing development and security of the ADSA.
        \section[Presidential Annual Report]
            \subsection The ASWWU President will construct a comprehensive summary of ASWWU's actions, accomplishments, and goals during their term in office. A physical and electronic copy will be delivered to the Alumni Center, Library Archives, and Student Life. Refer to Appendix C of the Personnel Manual for a listing of entities to be included.

    \article[Amendments]
        \section These Bylaws may be amended by a two-thirds vote of Senate and the signature of the President.

\title[Governance]

    \article[Executive Cabinet]
        \section[Membership]
            \subsection The officers listed below shall be members of the ASWWU Executive Cabinet and shall serve as advisers to the ASWWU President.
            \subsection The Executive Cabinet shall be as follows: President, ASWWU Chief of Staff, Executive Vice President (EVP), Social Vice President (SOVP), Spiritual Vice President (SIVP), Financial Vice President (FVP), Collegian Editor in Chief (CVP), Marketing Vice President (MVP), and the Portland ASWWU President.
            \subsection The ASWWU Sponsors shall serve as invitees to the ASWWU Executive Cabinet meetings.
            \subsection The Portland ASWWU President shall not be required to attend regular cabinet meetings.
        \section[Jurisdiction]
            \subsection The Executive Cabinet shall be charged with advising the ASWWU President on all matters pertaining to ASWWU. Executive Cabinet members shall executive the orders of the ASWWU President, except when other provisions are specifically made by the Constitution, Bylaws, Governing Documents, or resolutions by the ASWWU Members.
        \section[Procedure]
            \subsection Executive Cabinet shall meet on a weekly basis.
            \subsection Minutes of Executive Cabinet shall be kept by the ASWWU Chief of Staff.
        \section[Social Department]
            \subsection[Purpose]
                \paragraph The purpose of the Social Department is to unite the members of the ASWWU through social activities and entertainment under the supervision and authority of the ASWWU President.
            \subsection[Staff]
                \paragraph The SoVP, chosen as stipulated in the Elections Manual, shall direct the Social Department according to the guidelines of the Personnel Manual and the Governing Documents. The Social Department includes one administrative assistant, Tread Shed Manager, and Student Association Center Manager, who are subject to ratification by the Senate. 
                \paragraph The SoVP may remove employees through the processes outlined in the Personnel Manual
            \subsection[Duties]
                \paragraph The Social Department shall provide social stimulation to the members of the ASWWU by organizing and leading out in both major and minor activities throughout the school year.
        \section[Spiritual Department]
            \subsection[Purpose]
                \paragraph The purpose of the Spiritual Department is to unite the members of the ASWWU through religious activities and leadership under the supervision and authority of the ASWWU President.
            \subsection[Staff]
                \paragraph The Spiritual VP, chosen as stipulated in the Elections Manual, shall direct the Spiritual Department according to the guidelines of the Personnel Manual and the Governing Documents. The SVP may hire an administrative assistant who is subject to ratification of the Senate. 
                \paragraph The Spiritual VP may remove his/her administrative assistant through the process outlined in the Personnel Manual.
            \subsection[Duties]
                \paragraph The Spiritual Department shall provide spiritual support and leadership to the members of the ASWWU, as well as organize and/or lead out in spiritual events throughout the school year, such as Battleground, Student Week of Prayer, etc.
        \section[Portland Campus ASWWU]
            \subsection[Purpose]
                \paragraph The Portland Campus ASWWU shall act as an independently operating ASWWU but answer to the College Place Campus ASWWU’s Senate, Judiciary, and Governing Documents. 
            \subsection[Staff]
                \paragraph A Portland Campus ASWWU Financial Vice President shall be appointed by the Portland Campus ASWWU President
                \paragraph Other staff shall be appointed by the Portland ASWWU President as needed.
            \subsection[Duties]
                \paragraph The Portland Campus ASWWU shall serve its students by upholding the core mission deined in Title 1, Article 1.
        \section[Appointed Positions]
            \subsection[Appointment of the Financial Vice President]
                \paragraph The applications for the position of Financial VP shall be due the second Wednesday following the General Election.
                paragraph The incoming President, the outgoing President, and the current Financial VP shall review the applications, contact references, interview each of the candidates separately, and vote on an applicant.
                \paragraph The nominee shall be presented to Senate for ratification by a simple majority vote.
            \subsection[Appointment of the Marketing Vice President]
                \paragraph Applications for the position of MVP are due the second Wednesday following the General Election.
                \paragraph The incoming President, the outgoing President, and current MVP shall review the applications, contact references, interview each of the candidates separately, and vote on an applicant.
            \subsection Job descriptions for each of the members of Executive Cabinet shall be contained in the official ASWWU Personnel Manual.
        \section[Sponsors]
            \subsection One sponsor and at times a separate media representative are assigned to the ASWWU by Walla Walla University.
            \subsection[Sponsors are to:]
                \paragraph Be invited to Executive Cabinet, Officer, and department head meetings.
                \paragraph Review publications to ensure basic legal and university standards are met prior to being sent to print.
                \paragraph Be kept up to date regarding the general operations of the ASWWU.
            \subsection It is strongly advised that the ASWWU President have weekly scheduled meetings with each sponsor.

    \article[Executive Officers]
        \section[Membership]
            \subsection The officers listed below shall be members of the ASWWU Executive Cabinet and shall serve as advisers to the ASWWU President.
            \subsection The Executive Cabinet shall be as follows: President, ASWWU Chief of Staff, Executive Vice President (EVP), Social Vice President (SoVP), Spiritual Vice President (SiVP), Financial Vice President (FVP), Marketing Vice President (MVP), and the Portland ASWWU President.
            \subsection The ASWWU Sponsors shall serve as invitees to the ASWWU Executive Cabinet meetings.
            \subsection The Portland ASWWU President shall not be required to attend regular cabinet meetings.
        \section[Jurisdiction]
            \subsection The Executive Officers shall be charged with advising the ASWWU President on matters pertaining to their prospective departments. Members shall execute the orders of the ASWWU President, except when other provisions are specifically made by the Constitution, Bylaws, Governing Documents, or resolutions by the ASWWU members.
        \section[Procedure]
            \subsection An Executive Officers meeting shall meet at the discretion of the ASWWU President.
        \section Minutes of the Executive Officers meeting shall be kept by the ASWWU Chief of Staff.
        \section[Appointed Positions]
            \subsection The Executive Vice President shall be responsible for recommending Publication Editors and the Webmaster to the incoming President. The Senate shall ratify the Publication Editors and the Webmaster following their appointment by the incoming President.
            \subsection Nominations for appointed offices involving expenditures of ASWWU funds paid by the ASWWU shall not be effective until confirmed by the Senate.
            \subsection Nomination of the Parliamentarian, ASWWU Chief of Staff, Systems Manager, and SAC Manager shall be made by the incoming President before the final meeting of Senate preceding his term office.
            \subsection Job descriptions for each of the appointed officers shall be contained in the official ASWWU Personnel Manual.
        \section[Control]
            \subsection All candidates for the ASWWU Executive Office must undergo a formal GPA and compliance review through the Student Life Office before being permitted to run. Current officers will also be required to undergo a yearly review during winter quarter.

    \article[Legislative Branch: Senate]
        \section[Districts]
            \subsection The districts that shall be represented in the ASWWU Senate are:
                \paragraph District 1 --- Sittner \nth{1} \& \nth{2} Floor/Meske.
                \paragraph District 2 --- Sittner \nth{3} \& \nth{4} Floor
                \paragraph District 3 --- Conard
                \paragraph District 4 --- Foreman
                \paragraph District 5 --- Mountain View/Birch Apartments
                \paragraph District 6 --- Hallmark/Faculty Court/University-Owned Housing
                \paragraph District 7 --- Off-Campus
                \paragraph District 8 --- Portland
                \paragraph District 9 --- Faculty
                \paragraph District 10 --- Staff
            \subsection Districts shall be re-evaluated (but not necessarily changed) once every four years, starting in 2004, or more often if deemed necessary by Senate or the ASWWU President. 
            \subsection District 7 – Off-Campus shall include current student missionaries and ACA students in their representation, including them in weekly emails.
        \section[Membership]
            \subsection Each Senate district shall be guaranteed two seats as provided in the Constitution.
            \subsection Portland District shall elect two representatives from the Portland campus. If one or both seats is/are left vacant after Senate elections, the Portland campus students shall have two weeks to fill the vacancy/ies. If a Portland campus student is not found to fill the vacancy/ies, Portland students can elect representatives from the College Place campus.
            \subsection ASWWU Senators must be members of the ASWWU. The Senate in conjunction with district constituencies shall be the final judge of Senate membership. 
        \section[Terms of Office]
            \subsection The term of office of a Senator shall commence immediately upon the certification of his/her election and shall terminate upon the certification of the next election for that seat. Thus, a term of office shall commence in October and ordinarily end in the following October, except in the case of defeat by challenge, in which case the term of the seat in question would end upon certification of the challenger's election.
            \subsection Senators shall receive an hourly minimum wage, which may not be increased or decreased during their
            term of office.
        \section[Vacancies]
            \subsection The President of the Senate shall declare a vacancy when they certify that an election has failed to fill a seat or when a senator is impeached, removed from office, dies, or resigns.
            \subsection Notice of such vacancies shall be immediately published in The Collegian.
            \subsection When an eligible ASWWU member files a Declaration of Candidacy for a vacant seat, there shall be a selfstanding article or notice of filing printed in the next possible Collegian which shall include, but not be limited to, the candidate’s name and residential status. This duty shall be the responsibility of the President of the Senate.
            \subsection The ASWWU member that has filed shall fill the vacancy unless another eligible ASWWU member files for the same vacant seat within one week of the original filing or until four (4) days have completely expired after the issuance of the article/notice in The Collegian, whichever is longer.
            \subsection If a vacant seat is contested, an election (for that seat) shall be conducted within three (3) weeks of the close of the filing period, as stipulated in 4.4 and in accordance with the Elections Manual. 
        \section[Challenges]
            \subsection During the second full week of winter and spring quarters, any eligible member of the ASWWU may challenge a senator to an election by filing a Declaration of Candidacy for the senator's seat.
            \subsection The incumbent in a challenge shall not be required to file a Declaration of Candidacy.
            \subsection An announcement of the challenge period and (later) the names of any incumbents and challengers shall be published in the next The Collegian.
            \subsection The names of the incumbents and the challengers shall be placed on an official ballot for a vote of the constituents of the challenged seat within the first three (3) weeks of the quarter (in accordance with Section 3 and all other applicable provisions of this Article).
        \section[Senate Regulations]
            \subsection Senate shall be regulated by the Senate Procedural Rules
        \section[Senate Term Extension for Summer Quarter]
            \subsection In order to regulate ASWWU business transactions and employee hires that take place over the summer, the outgoing ASWWU President shall nominate one (1) senator from each district and extend their term through the Summer until the following Senate elections in the fall.
            \subsection The requirements of eligibility for the summer term extension shall be that the candidates are returning students in the next fall quarter and they are able to meet via Skype, conference calls, or other alternatives over the summer for Senate meetings.
            \subsection If there are not at least ten (10) eligible senators from the current Senate, then the outgoing ASWWU President may nominate senators from other districts to take their place. If after this step there are still not at least ten (10) eligible senators, the outgoing ASWWU President may nominate promising and willing students to fill the vacant seats.
            \subsection The Summer Senate Session shall collaborate with the incoming Executive Vice President on all Senate matters, shall run in the same manner as the session run during the school year, and shall abide by the Senate Procedural Rules, Roberts Rules of Order, and all other ASWWU Governing Documents. 

        \article[Judicial Branch]
        \section The judicial power of ASWWU shall be vested in one Supreme Court, whose decisions on the interpretation of the ASWWU Constitution and these Bylaws shall be binding on both the Executive and Legislative branches and whose membership shall be composed of three justices. The Chief Justice shall also be the Parliamentarian. The Justices shall be appointed by the President and ratified by the Senate, and shall serve for one school year beginning upon the day of graduation and serving until the following graduation.
        \section For their services, the judges (with the exception of the Parliamentarian, who shall not be doubly compensated) shall receive an hourly minimum wage, which may not be increased or decreased during their term of office.
        \section In order to perform their job properly, judges shall be sent copies of the minutes of all ASWWU meetings and shall be sent copies of all bills which are passed. If deemed unconstitutional, a bill may be vetoed and sent back to the Senate floor with recommendations.
        \section This body will meet when deemed necessary by the Chief Justice.
        \section The outgoing President shall nominate two judges for the following school year to be confirmed prior to the third meeting of the end of the school year. If the outgoing President fails to do so, the incoming President shall nominate two judges to be voted on by the Senate.

    \article[Committees and Boards]
        \section[Nomination Procedure]
            \subsection[Selection Process]
            \paragraph All nominations to committees and boards shall be made by the ASWWU President as outlined below.
            \paragraph The President must present nominees to Senate within three (3) weeks of the request for a nominee or within the first three (3) meetings of Senate if the nominations are to be made for existing committees described in the Walla Walla University Governance Handbook.
            \subsection[Approval to Walla Walla University and ASWWU Committees]
                \paragraph Senate shall review nominations from the President to determine the nominee’s fitness to hold the position according to the description of service provided by the nomination committee chair person.
                \paragraph The nominee shall not be required to attend any meetings of Senate unless the nominee is officially requested by at least two members of Senate. In the case that a nominee is required to attend, they must receive the request at least three (3) business days previous to the second reading of the bill addressing their nomination.
                \paragraph The WWU Nominating Committee chairperson shall be made aware of all nominees approved by Senate.
        \section[Elections Board]
            \subsection The Elections Board and all its functions shall be regulated by the Elections Manual.
        \section[Inclusive Committee]
            \subsection The Inclusive Committee will prioritize an atmosphere of inclusivity on the WWU campus identifying areas of concern within the WWU campus with regards to privilege and oppression.
            \subsection This Committee will defer consideration of suggestions, concerns, and ideas to the WWU Diversity Council and will report monthly summaries to the WWU Diversity Council.
            \subsection This Committee shall convene monthly and be chaired by the ASWWU President.
            \subsection The Committee will consist of the ASWWU EVP, Collegian in Chief, and eight other non-ASWWU student leaders on campus determined by the ASWWU cabinet.

    \article[Associated Manuals]
        \section[Elections Manual]
            \subsection All elections shall be governed by the Elections Manual.
            \subsection The Elections Manual may be amended by a two-thirds vote of the Senate.
        \section[Personnel Manual]
            \subsection All personnel shall be governed by the Personnel Manual.
            \subsection Elected officer job descriptions in the Personnel Manual may be amended by a two-thirds vote of the Senate.
            \subsection Hired job descriptions in the Personnel Manual may be amended by the ASWWU Executive Cabinet.
        \section[Judiciary Manual]
            \subsection Duties of the Justices are outlined by the Judiciary Manual.
            \subsection The Judiciary Manual may be amended by a two-thirds vote of the Senate.
        \section[Senate Procedural Rules]
            \subsection The Senate shall be governed by the Senate Procedural Rules.
            \subsection The Senate Procedural Rules may be amended by a two-thirds vote of the Senate.

    \article[Referendum and Recall]
        \section[Referendum]
            \subsection Referendum questions shall have the ability to appear on an official ASWWU ballot through one of two mechanisms, Senate-authorized legislation or the voter-initiative process.
            \subsection If the Senate wishes to place a referendum question on the ballot, a Governance Legislation bill shall be brought before the Senate for consideration with an explanation as to the rationale for the question and the exact language of the question that shall appear on the ballot. The Senate cannot place any commentary on that ballot that would unduly influence voter behavior.
            \subsection If the ASWWU members wish to place a referendum question on the ballot, a petition can be circulated stating the exact wording that would appear on the ballot. If twenty (20) percent of the ASWWU membership signs such petition, it shall be submitted to the Elections Board who shall place the question on the ballot. The ASWWU membership cannot place any commentary on that ballot that would unduly influence voter behavior.
            \subsection In no case does the Elections Board have the authority to modify, alter, or otherwise act upon above questions except to place them on the ballot.
        \section[Recall]
            \subsection Any ASWWU elected officer and the justices can be subject to a recall election. This right shall be exclusively reserved for the ASWWU membership and cannot be authorized by the ASWWU Student Senate.
            \subsection If the membership wishes to recall an ASWWU officer outlined in 2.1, the membership shall circulate a petition stating the officer(s) name(s) and position(s). No rationale for such petition is required. If thirty (30) percent of the ASWWU membership signs such petition, it shall be submitted to the Elections Board for action.
            \subsection If the Elections Board certifies that thirty (30) percent of the ASWWU membership has signed the above petition, it shall call a special election. The election must be called within two (2) weeks of the receipt of such petition and shall occur no later than four (4) weeks after receipt. The sole question on the ballot in a special election shall be a simple yes-or-no vote for removal of said officer(s). The ASWWU membership cannot place any commentary on that ballot that would unduly influence voter behavior.

\title[Financial Control]

    \article[Jurisdiction]
        \section The accounting systems and budgetary control of the ASWWU shall be controlled by the Senate, who holds the power of the purse.
        \section The Financial VP shall manage the accounting system and budget of the ASWWU under the guidance and leadership of the President.
        \section To transfer the control of a line item to another organization requires a two-thirds approval vote from the Senate and the signatures of the ASWWU President and the Executive VP. 

    \article[Budget Process]
        \section By 15 May or two weeks after the hiring of the Financial Vice President, whichever is later, the incoming ASWWU President, in consultation with other officers, shall present to the Senate for approval the budget for the following academic year.
            \subsection The Portland Campus Financial Vice President must present their budget within the same timeline.
        \section The budget shall contain planned, itemized expenditures and revenues for each department along with the previous two years’ budgeted expenditures and revenues for comparison. The master budget, which outlines the budget, shall include forecasted faculty, staff, and student dues, departmental net totals, required reserve account levels, and current reserve account levels. The master budget shall also contain the previous two years’ figures for comparison.
        \section The Financial VP, in preparing and maintaining the budget, shall ensure that “net income” will be budgeted at \$10,000. At no point will the FVP be able to prepare or present to Senate a budget that results in “net income” equaling less than this mandated amount.
        \section By the fourth regular meeting of the Senate, the approved ASWWU budget from the previous year (for the upcoming fiscal year) along with all proposed budget changes shall be presented to the Senate for approval. If revenue is lower than forecasted, the President and Financial VP shall propose line cuts and the list of prioritized budget needs. If revenue is higher than forecasted, each reserve must be updated first then expenditure lines can be adjusted upward limited to a percentage increase half that of the overall budget increase. Any changes from the proposed spring quarter budget should be explained and clarified upon approval of Senate.
        \section Department heads, with the approval of the President and Financial VP, are permitted to re-allocate budgeted amounts between their line items, so long as these changes do not result in a change ot that department’s effect on overall ASWWU net income. Any inter-departmental transfers, as well as any changes that result in an effect on overall ASWWU net income, shall be approved by Senate. 

    \article[Reserve Structure and Control]
        \section[Reserve Integrity]
            \subsection All changes affecting the balance of the ASWWU Budget must be entered by the Financial VP or Controller into the ASWWU Financial Records in the reserve section. The entry must include the date, the amount, the name and position of the person who is withdrawing/adding the money, and the reason the money was withdrawn/added. If a withdrawal is for the purchase of equipment or acquisition of some asset(s), the new asset(s) must be entered into the most recent end-of-the-year inventory for the department making the purchase 
        \section[Long-Term Projects Fund]
            \subsection A Long-Term Projects Fund shall be maintained by each ASWWU Administration.
            \subsection A yearly contribution to the Long-Term Projects Fund of \$20,000 is required of each ASWWU administration.
            \paragraph All contributions to the Long-Term Projects Fund must come from the overall ASWWU budget.
            \subsection The money in the Long-Term Projects Fund may not be used until it reaches a balance of \$80,000.
            \subsection The quarter before the Long-Term Projects Fund will reach \$80,000, the current ASWWU administration must start advertising for the fund. Official proposals may be presented once the fund reaches a balance of \$80,000.
            \subsection Any project which would be funded by the Long-Term Projects Fund must be presented in a proposal to the Long-Term Projects Fund Committee (LTPC). This committee would consist of the ASWWU President, Executive Vice President, Senate Finance Committee Chair, the Financial Vice President, the ASWWU Sponsor and each of the Senate Committee Chairs. This proposal must include:
                \paragraph Reasons section showing why project is needed on campus and how it will benefit students.
                \paragraph Area usage permission section granting usage of area on campus if needed in project from appropriate administrators.
                \paragraph Budget section on how funds will be used.
                \paragraph Analysis of different contracts section for fund spending and why current plan is most efficient.
                \paragraph Sustainability section outlining future operations of project.
            \subsection Following the first presentation of a proposal to the Long-Term Projects Fund Committee, a period of 10 school weeks must elapse during which other proposals may be submitted.
            \subsection As proposals approved by the LTPC, they will then be presented to University Master Planning Committee (UMPC) and their recommendations will be attached to the proposal.
            \subsection After all approved proposals have been reviewed by the UMPC, the LTPC will deliberate and select one of the approved proposals to bring to Senate.
            \subsection The winning proposal will be presented to Senate by the ASWWU president as new business. Senate will authorize disbursement of the Long-Term Projects Fund with a two thirds (2/3) vote.
        \section[Use of the ASWWU Emergency Fund]
            \subsection The ASWWU Emergency fund may be used when essential ASWWU services cannot be funded due to extraordinary events such as a large decrease in enrollment or large losses from theft or damage.
            \subsection In order to withdraw money from the Emergency fund, the President will write a financial legislation bill and submit that to a three-quarters approval vote of the Senate.
            \subsection A balance of \$40,000 must be maintained in the Emergency fund at all times.
            \subsection Funds withdrawn from the ASWWU Emergency fund for emergency purposes need not be replenished during the same school year. The resulting deficit in the Emergency fund shall be replenished to the requirement as stated in Section 4.3. by the following year’s budget as an operating expense.
        \section[Investment of Excess Funds]
            \subsection Excess funds available at the close of any given year’s budget on June 30 of that year, after the Quasi Endowment balance and Capital Reserve balance have been updated according to 2.1 and 4.3 of this Article, shall be referred to as net income. Any net income at year-end will be invested in the quasi endowment to grow the principal balance of the quasi endowment.
        \section[Atlas Lifecycle Fund]
            \subsection The Financial Vice President shall maintain a fund for lifecycle expenditures specific to the Atlas. These funds shall be available for major equipment purchases, repairs, and renovations.
            \subsection The Financial Vice President shall transfer \$2,000 into the Atlas Lifecycle Fund each academic year.
            \subsection Funds may be withdrawn by the Atlas Manager with the approval of Cabinet. The President shall note any withdrawals with the Senate. 
        \section[Feminine Products Lifecycle Fund]
            \subsection The Financial Vice President shall maintain a fund for lifecycle expenditures for feminine products. The funds shall be available for the purchase of feminine products and/or replacement dispensers.
            \subsection The Financial Vice President shall transfer \$1,200 into the Feminine Product Lifecycle Fund each academic year.
            \subsection Funds may be withdrawn by the ASWWU President with the approval of Cabinet. The President shall note any withdrawals with the Senate.

    \article[Financial Controls]
        \section The Senate Financial Committee shall maintain a financial log containing the budget and general reserve fund withdrawals/additions for each year. The continuity of this log is the responsibility of the acting President of the Senate in conjunction with the Senate Finance Committee. The financial log shall be titled ASWWU Financial Records and include the sections: 
            \subsection Yearly Budgets
            \subsection Departmental Inventories
            \subsection General Reserve Fund
        \section The Yearly Budgets section is to include a copy of the budget approved by the ASWWU Senate and actual expenditures corresponding to each budget for each school year. The approved budget is to be added to the financial records before the next regular session of the ASWWU Senate following its approval.
        \section The Departmental Inventories section is to include the end-of-the-year inventory for each ASWWU department. The incoming staff of each department must work with the outgoing staff of the same department to formulate an end-of-the-year inventory for that department. Inventory entries should include the make/model number of the item, approximate worth of the item, current condition of the item, and any special notes applying to the item. Items worth less than \$150 may be lumped together into one category entitled “miscellaneous.”
        \section The General Reserve Fund section is to contain the starting and ending balance of the General Reserve Fund (GRF) for each year. Any withdrawals from or additions to the GRF during a year shall be recorded between the year’s starting and ending balances. These withdrawal/addition entries shall list the amount withdrawn/added, the organization making the withdrawal/addition, and the reason for the withdrawal/addition.
        \section The Senate Finance Committee, in conjunction with the ASWWU Financial VP and the President, shall evaluate the need for an annual audit. If an audit is deemed necessary, it shall be appointed with the help of the administration of WWU as necessary.
        \section To ensure financial clarity and responsibility, the Financial VP shall review all non–credit card expenditures (which includes check requests, invoices, etc.). The Financial VP shall also review ASWWU credit card statements. 

    \article[Property]
        \section[Ownership of ASWWU Property]
            \subsection ASWWU is not incorporated; therefore, legal title and ownership of all property shall be vested in Walla Walla University, Incorporated. All financial transactions shall be subject to the regulations of that corporation. 
        \section[Use of Personal Property]
            \subsection Personal property that is being used by the ASWWU for official purposes may be borrowed upon contract by an ASWWU executive officer on behalf of the ASWWU. If the value of the item exceeds \$150, then the approval of the ASWWU President is required.
            \subsection No compensation may be made for the use of borrowed property.
            \subsection Items that are borrowed by the ASWWU shall be treated as ASWWU property for the duration of the contract and shall not be used for personal use.
            \subsection All borrowed items shall have a written contract that includes (1) the name of the owner; (2) the name and description of the item; (3) the serial number, if available; (4) the reason for ASWWU use; (5) the value of item in U.S. dollars; (6) the dates and times of the lease; and (7) the signatures of the owner and approving official.
        \section[Reimbursement for Damaged Property]
            \subsection If the borrowed item is damaged or lost by an ASWWU officer or employee, then the ASWWU shall reimburse up to full market value as determined by the ASWWU Financial VP. If a reimbursement exceeds \$150, Senate must be notified at the next Senate meeting. A report of the damage shall be filed with the Senate no later than the next regular Senate meeting.
            \subsection If personal property does not have a written agreement but is still used by ASWWU, then the owner shall take full responsibility for any damage or loss that may be incurred or caused by ASWWU officers, employees, or members.
        \section[Investigation of Damage]
            \subsection  If any ASWWU-owned or borrowed property is damaged, then investigation may be called for in accordance with Title 5, Article II.

    \article[Capital Purchases and Sales]
        \section Any department desiring to spend five hundred dollars (\$500) or more for capital expenditure items shall submit a request to Senate for approval.
        \section Any department desiring to sell one hundred and fifty dollars (\$150) or more worth of current capital shall submit a request to Senate for approval.
        \section If multiple items are planned to be purchased for a total of more than five hundred dollars (\$500), but the individual items do not exceed five hundred dollars (\$500), the total for the planned capital expenditure as a whole will be the number used when determining the need for Senate approval.
        \section Capital expenditures shall be defined as equipment or furniture with a reasonable life of at least one year.
        \section Expenditures over five hundred (\$500) dollars shall require the signature of the President and Financial VP. 

    \article[Department Inventories]
        \section By the first regular session of Senate and before the last regular senate meeting of spring quarter the incoming ASWWU President, in consultation with other officers, shall present to the Senate for review the end of the year inventories for the ASWWU Photo, Marketing, Video, and \textit{Collegian} departments.
        \section The accuracy of the departmental inventories shall be verified by both the outgoing and incoming departmental heads who shall sign the inventories before they are presented.
        \section Any discrepancies between the previous year's ending inventory and the current year's ending inventory shall be noted with an explanation by the outgoing head of the department.

    \article[Contracts and Long-Term Purchasing Orders]
        \section Contracts and long-term purchasing orders shall be signed by the President.

    \article[Hiring and Contracts]
        \section[Hiring]
            \subsection All ASWWU positions shall be announced in The Collegian and similar media two weeks before applications are due so as to ensure equal access and fairness in selection.
            \subsection All non-elected positions, whether in a paid or unpaid role, shall be required to submit to the appropriate branch an ASWWU Personnel Application and a résumé. All employee documentation will be kept on record in their employee file in the executive offices.
            \subsection Employees may be paid for work completed during seven (7) consecutive days prior to the employment authorization. Work completed prior to this period will be regarded as unpaid volunteer time. It is the responsibility of the ASWWU to inform its employees of this practice.
            \subsection For all persons requiring Senate confirmation, a copy of above documents shall be included with bill for consideration by the Senate. Upon selection of candidate, the person shall be given a copy of all job related ASWWU governing documents, including but not limited to, the Constitution, Bylaws, and Personnel Manuals.
            \subsection Hiring procedures are further directed by the Personnel Manual
        \section[Contracts]
            \subsection Any time official funds are used for a project in excess of one thousand dollars (\$1,000), the ASWWU shall open project bidding to the public. The public shall be made aware of such projects through The Collegian and other available media. The available bidding time for projects shall last for no less than two (2) weeks.
            \subsection The bid shall include, but not be limited to, a project proposal, detailed cost estimations, a detailed timeline, and other pertinent documents as requested.
            \subsection Should an ASWWU employee, in any capacity, have a fiduciary or other conflict of interest with the party bidding for such projects, the employee should send written notification of such conflicts to the President, Parliamentarian, and Senate Finance Committee chair. The employee then must remove themselves from any situation in which the said project might arise. If an employee does not disclose such conflicts, the contract is declared null and void immediately and without recourse.
            \subsection Copies of all such contracts shall be made available to the Senate upon request.
            \subsection Contracts lasting longer than the current fiscal year must be approved by a two-thirds vote of the Senate.

    \article[ASWWU Credit Cards]
        \section The President, Social VP, and Financial VP shall each have their own ASWWU credit card in order to streamline the process of making ASWWU purchases.
        \section Credit card statements shall be reviewed by the Financial VP and the chairperson of the Senate Financial Committee as stipulated in Title 3, Article IV, line 6 of the ASWWU Bylaws.

    \title[Publications]

    \article[Oversight]
        \section The Marketing VP shall be responsible for all the ASWWU publications except \textit{The Collegian}.
        \section \textit{The Collegian} Editor-in-Chief shall be responsible for \textit{The Collegian}.

    \article[The Collegian]
        \section[Purpose]
            \subsection \textit{The Collegian} is the official newspaper of the ASWWU.
        \section[Staff]
            \subsection \textit{The Collegian} shall conduct its business under the supervision of the Editor-in-Chief.
            \subsection \textit{The Collegian} Editor-in-Chief shall be ratified by the Senate.
            \subsection The Editor-in-Chief shall work to fill his/her staff, which shall include page editors, designers, writers, an office assistant, a photo editor, and other members, as the Editor-in-Chief deems necessary, to fulfill the duties of business.
            \subsection Prospective candidates shall meet the qualifications of senators and shall submit applications to the Editor.
        \section[Duties]
            \subsection Publish a weekly paper for every week of autumn, winter, and spring quarters, not including dead week and finals week.
            \subsection Document historically important events on campus, current news events, and items of entertainment value.
            \subsection The Editor-in-Chief shall be responsible for meeting publication dates, for meeting copy deadlines with the ASWWU Sponsor, for formulating editorial policies, and for protecting the ASWWU against legal liability in The Collegian.
            \subsection The Editor-in-Chief of The Collegian shall also be responsible for the weekly delivery of The Collegian to the Portland campus. 

    \article[Mountain Ash]
        \section[Purpose]
            \subsection The \textit{Mountain Ash} is the annual of the ASWWU and shall portray WWU’s year of activities. As such it shall give a full and representative view of the university and its activities, shall serve as a memorial for students who have attended school, and shall seek to make the school attractive to its constituency.
        \section[Staff]
            \subsection The \textit{Mountain Ash} shall conduct its business under the supervision of the Editor, who is supervised by the
            Marketing Vice President and under general supervision of the ASWWU as exercised through the Senate. Or, the Mountain Ash can be done by an outside source if approved by a majority vote of the Senate.
            \subsection The Editor shall work to fill his/her staff, which shall include designers, a photo editor, writers, and other members, as the Editor deems necessary, to fulfill the duties of business.
            \subsection Prospective candidates shall meet the qualifications required of senators and shall submit applications to the Editor.
        \section[Duties]
            \subsection The Editor shall be responsible for meeting publication dates, for meeting copy deadlines with the advisor and the press, and for formulating editorial policies and shall protect the ASWWU against legal liability in the \textit{Mountain Ash}.

    \article[The Mask]
        \section[Purpose]
            \subsection \textit{The Mask} is the directory of WWU, containing photos of students, staff, and faculty.
        \section[Staff]
            \subsection \textit{The Mask} shall conduct its business under the supervision of the Editor who is supervised by the Marketing Vice President and under general supervision of the Association as exercised through the Senate.
            \subsection \textit{The Mask} Editor shall ratified by the Senate.
            \subsection The Editor shall work to fill his/her staff which shall include page editors, designers, writers, an office assistant, a photo editor, and other members, as the Editor deems necessary, to fulfill the duties of business.
            \subsection Prospective candidates shall meet the qualifications required of senators and shall submit applications to the Editor.
            \subsection The duty of staff dismissal shall fall to the Executive Officers.
        \section[Duties]
            \subsection The Editor shall be responsible for meeting publication dates, for meeting copy deadlines with the advisor and the press, and for formulating editorial policies and shall protect the ASWWU against legal liability in \textit{The Mask}.
        \section[Distribution]
            \subsection The Editor shall coordinate the distribution of The Mask to members at all WWU campuses.
            \subsection If a \textit{Mask} is not physically distributed, an online version must be made available to all campuses.
            \subsection The Editor shall be responsible for setting the price at which the community, WWU administration, and
            departments may purchase copies of \textit{The Mask}. 

\title[Jurisprudence]

    \article[Compliance]
        \section All members and employees of ASWWU serving in any capacity whatsoever within the ASWWU must comply with the Constitution, Bylaws, Manuals, and all other ASWWU Governing Documents.
        \section Willful and/or deliberate disregard of these Bylaws shall be grounds for removal from office through impeachment, recall, or dismissal. Not being aware of such documents and regulations shall not be accepted as a defense.

    \article[Investigations and Misconduct]
        \section[Initiation of an Investigation]
            \subsection At the request of any two ASWWU Executive Cabinet, Judicial, or Senate voting members, or as called for elsewhere in the ASWWU governing documents, an official investigation shall be conducted by the Senate Personnel Committee in consultation with the Senate Governance Committee. At the discretion of the Personnel Committee Chair and the Chief Justice, the investigation may be deferred to the Supreme Court.
        \section[Reports]
            \subsection A report of the investigation shall include any and all findings and a recommendation regarding further action, if needed. The report shall be presented and filed with the Senate, the ASWWU President, and the Supreme Court no later than five (5) academic days from the close of the investigation.
        \section[Actions]
            \subsection At least five (5) academic days prior to the Senate hearing, the defendant will be informed in writing of the time and date of the said hearing and be requested to attend. Following the hearing the defendant will be given written notice of all actions within five (5) academic days. Actions available for dealing with misconduct or any other investigated offense includes but is not limited to:
                \paragraph Removal from office (see Article III, Section 5 of the ASWWU Constitution).
                \paragraph Censure shall include a temporary suspension of privileges of office, voting rights, and pay for no more than ten (10) academic days. The hearing shall follow the same procedure as for impeachment.
                \paragraph In cases involving destruction or damage of property, fines may be levied against those involved, not to exceed the cost of replacement or repair. A two-thirds vote of the Senate is required to levy a fine.
                \paragraph A written reprimand sent to the reprimanded by the Senate. A reprimand shall not include any suspension of privileges or rights. A majority vote of the Senate is required to issue a reprimand.
        \section[Appeals]
            \subsection All actions, except for impeachment, may be appealed to the ASWWU Supreme Court. If overturned, the case shall be referred back to the Senate with recommendations.
        \section[Removal of a Senator]
            \subsection Senators of the ASWWU are to be held to a higher standard than other ASWWU members. In such case an ASWWU senator is placed on citizenship probation by the WWU Administration the following procedures are to be enacted:
                \paragraph The senator will be presented with a waiver of confidentiality by the Vice President for Student Life pertaining to the circumstance for which he/she has been placed on probation by the WWU Administration. If the senator does not wish to waive his/her confidentiality, the senator must resign his/her seat.
                \paragraph If the senator chooses not to resign their seat, the following steps will be enacted and the ASWWU Executive Vice President will be notified. The Senate will be presented with the facts pertaining to the senator’s probation by the applicable administration officials. The senator will also be given the opportunity to speak for ten (10) minutes in their own defense. Afterward, the Senate will vote anonymously whether or not to dismiss the senator in question. If the senator is not dismissed and they hold an office within the senate (e.g. President Pro Tempore or committee chair), an anonymous vote to dismiss the senator from that position will also be taken.
                \paragraph If the Senate does not dismiss the senator, the facts pertaining to the senator’s probation will be distributed in writing to that senator’s district along with the procedure by which the constituents within that district may dismiss the senator if they so choose.
                \paragraph If neither the ASWWU Senate nor the senator’s district chooses to dismiss the senator in question, no other body or office will be given the power by which to affect the senator’s status within the ASWWU Senate.

    \article[Conflicts of Interest]
        \section[Concurrent Positions]
            \subsection No person shall hold two appointed or elective offices, including that of senator, within the ASWWU concurrently without permission granted by a two-thirds vote of the Senate. Senators holding a paid position and approved concurrently shall recuse themselves from any vote, within reason, affecting their concurrently held position.
            \subsection No person shall hold a position with voting privileges in more than one of the three ASWWU branches: Executive, Legislative, and Judicial.
        \section[Assets and Influence]
            \subsection In any situation involving ASWWU assets or opportunity to influence the ASWWU unduly where there is the appearance of a conflict of interest, an investigation by the Senate or Judicial Branch is required. A report of the investigation shall include any and all findings and a recommendation regarding further action, if needed. The report shall be presented and filed with the Senate, ASWWU President, and Judges, no later than fourteen (14) days from the close of the investigation.
            \subsection In the event of unavoidable conflicts of interest, approval must be obtained through Senate by a two-thirds vote.

    \article[Security Systems]
        \section[Ownership and Placement]
            \subsection All security cameras being used to monitor ASWWU facilities are to be installed or uninstalled by request of the ASWWU President and majority vote of the ASWWU Senate.
        \section[Use of Footage and Monitoring]
            \subsection No footage taken by ASWWU security cameras is to be made public, unless written requirement is made by law. Exceptions will require written permission of all recorded parties, the ASWWU President, and the ASWWU Executive VP.
            \subsection In order to view footage taken by the cameras, authorization must be given by the ASWWU President and Executive VP.
            \subsection Campus Security is authorized to use monitoring capabilities for theft prevention and to protect against imminent danger. It is to be otherwise assumed that persons in the ASWWU facilities who appear to have made legal entrance are there with correct authorization.
            \subsection Should Campus Security have other concerns based on having monitored activities via ASWWU security cameras, they should first speak with the ASWWU President or Executive VP before taking any other action. 

\end{document}
