\input{Formatting/Judiciary_Manual-Preamble.tex}

\article[Authority]
The Judiciary Manual shall be the authority for the ASWWU Judiciary and shall not override the ASWWU Constitution or its Bylaws. All Judiciary meetings shall be conducted according to the Judiciary Manual Procedures and \textit{Robert’s Rules of Order}.

\article[Members]
    \section[Selection]
        \subsection The Chief Justice shall be nominated by the outgoing ASWWU President for the following school year and shall be confirmed in Senate prior to the third meeting of the end of the school year. If the outgoing President fails to do so, the incoming President shall nominate the Chief Justice to be approved by the Senate.
        \subsection The two (2) Justices shall be nominated by the outgoing President for the following school year to be confirmed in Senate prior to the third meeting of the end of the school year. If the outgoing President fails to do so, the incoming President shall nominate the two (2) Justices be approved by the Senate.
        \subsection The vote margin for Senate confirmation shall be a two-thirds vote.
        \subsection The qualifications for members of the Judiciary shall be the same as for Senators. All Justices shall abide by the regulations found in the ASWWU Personnel and Judiciary Manuals.

    \section [Duties]
        \subsection The Chief Justice shall preside over all Judiciary meetings.
        \subsection The Judiciary will meet when deemed necessary by the Chief Justice. Additional meetings may be held as necessary, or by request of the President or the Senate.
        \subsection The Judiciary shall be responsible for deciding and interpreting all cases regarding ASWWU activities personnel, or governing documents brought before it.
        \subsection Minutes shall be taken during the Judiciary meetings by one of the three Justices but cannot be released to any outside entity prior to the decision of the case before the body. 

    \section[Recall]
       \subsection The Justices shall be subject to recall petition or removal per either Bylaws Article VII, Section 2, or the Personnel Manual.

\article[Authority and Procedures]
    \section[Authority]
        \subsection The power to act shall include, but not be limited to:
            \paragraph Taking the complaint under advisement or rendering an opinion on any business regarding the ASWWU, its Governing Documents, or its personnel;
            \paragraph Reconsidering, upon petition of either party involved with the complaint, a prior act if at least two Justices deem it appropriate;
            \paragraph Validating or invalidating all ASWWU elections if election procedures were not followed or it is found that fraud or other election breaches occurred;
            \paragraph Establishing by a majority vote Judicial Board procedures and guidelines for the hearing of cases.

    \section[Procedures]
        \subsection A party shall be able to file a petition in a manner prescribed by the Judiciary Manual, so long as that complaint involves some aspect of the ASWWU.
        \subsection The petition must include all of the following: the party or parties in question, the background for the petition, the reason for the petition, and a question that seeks solution in some manner.
        \subsection Upon receipt of a properly filed petition, the Judiciary shall act within ten (10) school days of the formal filing of a petition with the Chief Justice.
        \subsection The Chief Justice shall then notify all parties of the date, time, and location of the hearing.
        \subsection If necessary, defendants shall be notified in writing of the charges against them at least five (5) school days before the hearing.
        \subsection The ASWWU Chief of Staff and/or Senate Secretary shall take minutes during the oral arguments.
        \subsection Upon decision of the petition, the Chief Justice shall inform in writing within three (3) school days the parties involved, the Executive Branch, and the Senate of the decision, which shall be entered into the ASWWU record.

\article[Amendments]
    \section Amendments to the Judiciary Manual shall be made by a two-thirds vote of the Senate.

\end{document}