\input{Formatting/Senate_Procedural_Rules-Preamble.tex}

\article[Scope]
This document shall govern the procedures used by the Associated Students of Walla Walla University (hereafter referred to as the ASWWU) Student Senate. Only the ASWWU Constitution and Bylaws and applicable local, state, and federal law shall supersede it. The current edition of \textit{Robert’s Rules of Order} shall govern any matter of procedure not contained within this document.

\article[Organizational Structure]

    \section[Definitions]
        \subsection Majority: More than half of the members present minus any abstentions.
        \subsection Two-thirds vote: A two-thirds vote of total votes cast either for or opposed. This is two-thirds of the members present minus any abstentions.
        \subsection Senate: The Student Senate of the ASWWU
        \subsection Senate Executive Committee: Consists of the President of the Senate, the chairs of the Senate standing committees, the Senate President Pro-Tempore, and the ASWWU Parliamentarian.
        \subsection Chair: In the case of the full Senate, the ASWWU Executive Vice President (EVP). In the case of the committees, the chair of the committee. In cases where the usual chair is not present or has temporarily passed the gavel to a successor, the term chair shall refer to the successor.
        \subsection Quorum: Two-thirds the number of filled Senate seats, or 10, whichever is less.
        \subsection Senate Posting Locations: Designated bulletin boards in the Student Association Center, all dormitories, senators' email addresses, \textit{The Collegian}, and the ASWWU website. Official Senate documents are to be sent to each senator's Walla Walla University email address unless a mutual agreement, with the knowledge of the President of the Senate, is made between the senator and the Assistant to the EVP.

    \section[Senate Officers]
        \subsection Senate shall appoint by majority vote a President Pro-Tempore at the second regular meeting after each regular Senate election.
        \subsection The Executive VP of the ASWWU shall serve as the President of the Senate.
        \subsection The Senate Secretary, as according to their job description, shall record the official minutes of the Senate and shall maintain a permanent record of all Senate business. He/she shall send copies of the minutes after each Senate meeting to the Executive VP. He/she is responsible for sending at least an electronic copy of finalized official Senate business to the E.L. Mabley Archives. He/she is also responsible for coordinating the revision of the Senate Procedural Rules as needed.
        \subsection The ASWWU Parliamentarian shall advise the Senate of matters of parliamentary procedures and questions concerning the Constitution or Bylaws. He/she shall be responsible for coordinating the revision of the Constitution and Bylaws as needed.
        \subsection The Senate President Pro-Tempore shall assume the leadership role of Senate if the EVP is tardy seven (7) or more minutes or convenes Senate seven (7) or more minutes after the official starting time. If the EVP is tardy seven (7) or more minutes or convenes Senate seven (7) or more minutes after the official starting time three (3) or more times per academic quarter, he/she shall be subject to disciplinary action by the ASWWU President, as approved by the appropriate committee.
        \subsection In case of the removal of the EVP from office, his/her death, resignation, or inability to discharge powers and duties of the said office, the same shall devolve on the Senate President Pro-Tempore. The Senate at the next regular meeting shall elect a new President Pro-Tempore.
        \subsection In case of the removal of the Senate President Pro-Tempore from office, his/her death, resignation, or inability to discharge the duties of the said office, the Senate shall elect a new Senate President ProTempore at the next regular meeting.
        \subsection In case of the removal of any elected VP from office, his/her death, resignation, or inability to discharge the powers of the said office, the ASWWU President shall appoint a person to the said office, subject to ratification by a two-thirds majority of the Senate.
        \subsection Should a bill be brought before Senate for removal of an elected officer, the Governance Committee will serve as a trial committee per \textit{Robert's Rules of Order}. Once the procedures outlined are completed, a special session shall be called for the purpose of removing the officer. 

    \section[Senate Committees]
        \subsection The Senate shall contain at least four (4) standing committees, which shall be established each school year at the first Senate session of the year. They shall include the following: 
            \paragraph[Finance Committee:] This committee shall review any financial legislation (F.L.) bill that is brought to Senate. The chair of the committee shall meet with the ASWWU Financial VP (FVP) according to the bylaws to approve ASWWU purchases, statements, and check requests
            \paragraph[Governance Committee:] A committee to deal with any aspect of the ASWWU Constitution and Bylaws, this document, or other governance issues. This includes updating the governing documents as needed.
            \paragraph[Personnel Committee:] A committee to deal with the implementation of proper personnel procedure, excluding the process of hiring, as stipulated by the ASWWU Personnel Manual. All positions not included in the Personnel Manual shall be subject to the Governance Legislation process. Actions of this committee must be in accordance with WWU policy, Washington state law, and federal law.
            \paragraph[Review Committee:] A committee to deal with representing views concerning issues outside the realm of ASWWU executive power. Specifically, this committee shall be tasked with reviewing University policy at large and ensuring that any concern of the ASWWU pertaining to University policy be heard by the relevant powers. This committee shall review any Senate resolution (S.R.) bill that is brought to Senate. The chair of the committee shall meet with the Vice President of one of the four principal divisions of University Administration (Student Life, Academic Administration, Financial Administration, or University Relations and Advancement) to which the bill concerns and reach a consensus with the VP regarding expected response protocol to S.R. bills for the academic year.
        \subsection Ad hoc committees may be established as needed by a majority Senate vote
        \subsection The method for determination of chairs for standing committee shall be by vote of the members of each committee at the first Senate session
        \subsection Standing committees shall contain one Senator per on-campus student District (Districts 1-7). Senators within the same District will be allowed to discuss and choose which standing committees they will join. One of the Senators must be in each committee, and both Senators mmust be in at least one committee, but membership may be unevenly split between District partners. Senators may switch any or all their committees at the first Senate meeting of winter and spring quarters with approval from their District partner and without a committee vote. Senators may switch committees at any other point in the quarter with partner approval and a two-thirds vote of the committee. The EVP may adjust committee membership and District representation to increase the efficacy of committees with a majority vote of Senate. 
        \subsection For the efficiency and effectiveness of all committees, senators will be allowed to choose in which committee they would like to serve. In the event where the initial committees are uneven, the EVP can redistribute the senators
        \subsection Senators representing the Faculty, Staff, and Portland Districts are not required to sit on a Senate Committee. 

\article[Working Policy]

    \section[Attendance]
        \subsection Regular attendance is required of senators. Two (2) absences, excused or unexcused, per quarter are allowed. Upon a third absence, the senator shall be required to appeal to the ASWWU Executive Committee. The ASWWU Executive Committee shall determine whether or not the senator is to be removed from Senate.
        \subsection Anyone who is seven (7) or more minutes late from the time the meeting was scheduled to start is considered tardy. Two (2) unexcused tardies equal one (1) absence, unless the senator has made previous arrangements with the Senate Executive Committee, the EVP, or the Assistant to the EVP. Senators who are tardy (unexcused) shall be denied the privilege of voting for that Senate session.
        \subsection Senators who use technology or are working on non-Senate related tasks during Senate will be counted as tardy. Proper use of technology involves viewing Senate documents and researching information pertinent to the current matter being discussed. The enforcement of this policy shall be made verbal and shall be left up to the discretion of the EVP.
        \subsection A roll call shall be held at the beginning of the meeting. The secretary at the end of the meeting shall check attendance visually. A senator must be marked present at both checks to be considered present for that meeting. Roll calls may also be taken when a senator suggests the absence of quorum.

    \section[Legislation]
        \subsection All bills must be sponsored by one or more senators, the EVP, the ASWWU President, or the ASWWU President-Elect.
        \subsection Bills shall be typed with title, preamble, action, vote margin required to pass, contingency if the requirements of the bill are not met, a projected date of completion, and a clearly defined location at the end of the bill for the signature of the Chair of the appropriate committee, President of the Senate, and the President.
        \subsection Upon the date of completion given in the preamble, the benefactor shall return to Senate and present evidence of the bill's implementation. Upon request, the bill's sponsor shall give progress reports. 
        \subsection Before bills are considered in a Senate meeting, they shall be submitted, in advance, to the Assistant to the EVP. The Assistant to the EVP shall review each bill for correctness based on a pre-established format. If the format of the bill in question is incorrect, he/she must suggest corrections to the author of the bill as soon as possible so that it can be corrected before consideration. Any bill meeting correct format will be placed on the agenda.
        \subsection If such suggestions are not followed, the bill can nonetheless be submitted by an amendment to the agenda per \textit{Robert's Rules of Order}.
        \subsection The author of the bill, with the approval of the Assistant to the EVP, shall classify all pieces of legislation into one of the following categories:
            \paragraph[Financial Legislation (F.L.):] A bill which deals with any aspect of ASWWU funding and/or expenditures. Such legislation will, if deemed necessary, be referred to the Finance Committee. If the bill pertains to a revision of the ASWWU Governing Documents, the bill must be brought as a G.L. bill but must be considered by both committees in a joint committee hearing.
            \paragraph[Governance Legislation (G.L.):] A bill which deals with any aspect of the ASWWU Governing Documents. When changing the ASWWU Bylaws, a two-thirds vote margin of Senators present is required. When changing the Senate Procedural Rules, a resolution format with a G.L. designation will be used since a normal bill format would allow the ASWWU President to veto internal Senate rules.
            \paragraph[Senate Resolution (S.R.):] A bill that expresses the official opinion of the Senate on a particular issue that doesn't necessarily require ASWWU to take any action. All resolutions require a two-thirds vote margin. S.R. bills may be addressed to University Administration concerning University policy or practice, and shall either suggest alteration, request explanation, commend policy/practice, or criticize policy/practice. Any S.R. bills addressing the University administration shall specifically address the Vice President of one of the four principal divisions of University Administration (Student Life, Academic Administration, Financial Administration, or University Relations and Advancement). The VP addressed in an S.R. bill shall respond to the issues underlined in the bill in accordance with the protocol mutually agreed upon by their own department and with the chair of the Review Committee as underlined in Article II, Section 3.1.4.
            \paragraph[Personnel Legislation (P.L):] A bill that deals with the hiring of ASWWU employees or the implementation of proper procedure throughout the hiring process.
        \subsection After bill classification and the approval of bill format, the Assistant to the EVP shall make sure that copies are sent to all current senators' email addresses. Paper copies should be available upon request. At the next scheduled meeting, the bill will get its first reading during New Business. No debate will take place after the first reading in New Business until open forum of the current meeting.
        \subsection All bills must be referred to the appropriate committee for action between the first and second readings. This can include amendments and defeat of the bill.
        \subsection Vote tallies and any amendments must be given to the senators for consideration at the second meeting.
        \subsection At the next Senate meeting, the sponsor(s) of the bill will be given an authorship speech of up to five (5) minutes. After the sponsor(s) has been given the chance to speak, the chair of the proper committee shall update Senate on the status of the bill, after which normal debate may begin.

    \section[Senate Sessions]
        \subsection The Senate shall conduct its regular meetings each week at a time determined by a Senate resolution. Only this Senate Resolution shall officially authorize a Senate regular meeting.
        \subsection The agenda for a regular meeting of the Senate shall be posted for at least twenty-four (24) hours before the meeting at select Senate Posting Locations. Items not on the agenda must be placed on the agenda through an amendment vote on the agenda.
        \subsection The minutes of all Senate business shall be placed in select Senate Posting Locations before the next Senate meeting.
        \subsection The Senate shall not hold regular sessions during finals week of any quarter.
        \subsection Special sessions shall be announced at least one (1) day in advance. Announcements shall be posted at Senate Posting Locations.
        \subsection All Senate meetings shall be open to any members of the ASWWU, unless executive session is invoked in accordance with the procedures outlined in \textit{Robert's Rules of Order}.
        \subsection Voting may be by any parliamentary method outlined in the most recent edition of \textit{Robert's Rules of Order} at the discretion of the Senate President and as mentioned in the Constitution. 
        \subsection In the case of a tie, the Senate President may cast the deciding vote.
        \subsection Total tallies of votes shall become part of the official minutes of the Senate.
        \subsection For all purposes, a quorum to do business shall consist of two-thirds the number of filled Senate seats or ten (10) senators, whichever is less. Filled Senate seats are those to which a Senator has been elected and which have not been vacated by death, resignation, expulsion due to absence, or other means.
        \subsection The Senate President, as well as a majority vote by Senate, has the authority to establish a pre-determined time limit on the discussion of a particular bill or issue. When this time limit is reached, senators may extend debate time by a two-thirds vote.
        \subsection All paperwork pertaining to Senate, including the governing documents, bills, agendas, and minutes, shall be made available in print within one (1) week to any senator upon request. Senators shall have the ability to print and copy a reasonable number of copies in the ASWWU offices for senatorial business.
        \subsection Each week an update of the ASWWU Senate budget shall be presented. 

    \section[Committee Meetings]
        \subsection Committee meetings must occur whenever business pertaining to the committee is on the agenda. Additionally, committee meetings can be called at the discretion of the President of the Senate, the committee chair, or a two-thirds vote of the committee.
        \subsection All business that occurs must be done in concurrence with the ASWWU governing documents.
        \subsection If the Senate Secretary cannot attend a committee meeting, the committee is responsible for either appointing their own secretary or digitally recording the meeting. The recording or minutes shall then be emailed to the Senate Secretary.
        \subsection A committee shall have the right to modify or otherwise alter the language of a bill, the preamble notwithstanding, vote to table the bill for a week, or vote to permanently table a bill. However, should a committee vote to permanently table a bill, a two-thirds discharge vote can be brought by the Senate requiring that the bill come before the Senate, in old business, for a second reading. If Senate votes on the bill, it must pass by a two-thirds vote.
        \subsection No bill can come for a second reading that has not been sent to a committee meeting for action prior to the reading, with the exception provided in part 4.4
        \subsection Each committee must keep an updated list of the bills it reviews. This list must comply with the following form:
            \begin{center}
            \begin{tabular}{|l|l|l|l|}
            \hline
            Bill & New Business & Old Business & Progress Notes \\ \hline
            F.L. 1 & X & Passed & Money transferred to department X\\ \hline
            F.L. 2 & X & Did not pass & \\ \hline
            F.L. 3 & & & \\ \hline
            \end{tabular}
            \end{center}
        \subsection At the end of each quarter, an up-to-date list of all considered bills must be submitted to the EVP.

\article[Ratification and Amendment]

    \section This document and any amendments to it must be ratified by a two-thirds vote of senators present.


\end{document}
